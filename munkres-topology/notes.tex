\documentclass{report}

\usepackage{amsmath}
\usepackage{amsthm}
\usepackage{amssymb}

\usepackage{xspace}
\usepackage{enumitem}


\newtheorem*{remark}{Remark}
\theoremstyle{definition}
\newtheorem{definition}{Definition}[chapter]
\newtheorem{example}{Example}[definition]
\newtheorem{property}{Property}[chapter]

\newcommand{\defn}[1]{\textbf{#1}\label{#1}}
\newcommand{\set}[1]{\ensuremath\{#1\}}

\newcommand{\ie}{\textit{i.e.}\xspace}
\newcommand{\aka}{\textit{a.k.a.}\xspace}
\newcommand{\eg}{\textit{e.g.}\xspace}

\newcommand{\RA}{\ensuremath\Rightarrow}

\newcommand{\ZZ}{\mathbb{Z}}
\newcommand{\RR}{\mathbb{R}}

\newcommand{\A}{$A$\xspace}
\newcommand{\B}{$B$\xspace}
\newcommand{\X}{$X$\xspace}
\renewcommand{\a}{$a$\xspace}
\renewcommand{\b}{$b$\xspace}
\newcommand{\x}{$x$\xspace}

\newcommand{\ex}{\ensuremath\exists}

\begin{document}

\title{Notes for James R. Munkres' Topology (2E)}
\author{Shou}
\date{\today}

\maketitle
\tableofcontents

\setcounter{chapter}{-1}
\begin{chapter}{Structure and reading plans}
Ch 1-8 is the part I, mainly for common topology. The part II includes
ch 9-14, that depends on ch 1-4, is about algebraic topology.

My plan is to read through ch 1-4 very quickly, within a weekend, and
then I will start reading ch 9+ simultaneously with W.S.Massey's
Agelbraic topology: An induction.

Finally I wish I could finish all ch 1-8 and also some parts after ch 9.
\end{chapter}

\begin{chapter}{Set theory and logic}
  \begin{definition}{\defn{Order relation}}
    rel $C$ on set $A$ is called \emph{order relation} if
    \begin{enumerate}
    \item comparability, \ie
      $\forall x,y\in A, x\not=y\RA x C y\lor y C x$
    \item non-refl, \ie $\forall x, \neg(x C x)$
    \item trans, \ie $\forall x C y\land y C z, x C z$
    \end{enumerate}
    (\aka linear order)
  \end{definition}

  \begin{definition}{\defn{Open interval}}
    if $X$ is a set and $<$ is an order rel, and if $a<b$ we use
    notation $(a,b)$ to denote $\set{x\in X\mid a<x<b}$, called
    \emph{open interval}.

    If $(a,b) = \emptyset$, then $a$ is called \emph{immediate
      precessor} of $b$ and $b$ called \emph{immediate successor} of
    $a$.
  \end{definition}
  \begin{remark}
    It makes more sense on $X$ is a discrete set. Since if
    $(a,b)$ is an open interval in $\RR$, $(a,b) = \emptyset
    \RA a = b$ which makes no sense on $a$ as an immediate precessor
    of $b$.
  \end{remark}

  \begin{definition}{\defn{Order Type}}
    if $A$ and $B$ are two sets with $<_A$ and $<_B$. We say that \A
    and \B have same \emph{order type} if $\ex f : A \to B$ that
    preserves order, \ie $$a_1 <_A b_1 \RA f(a_1) <_B f(b_1)$$
  \end{definition}
  \begin{remark}
    It's just a generalization of monotone function.
  \end{remark}

  \begin{definition}{\defn{Dictionary order relation}}
    if \A,\B are two sets with $(<_A,<_B)$, defn an order for
    $A\times B$ by defining $$a_1\times b_1 < a_2\times b_2$$
    if $a_1<_A a_2$, or if $a_1=a_2\;\land\;b_1<_B b_2$.
  \end{definition}

  \begin{definition}{\defn{LUB property}/\defn{GLB property}}
    For $A$ and $<_A$, we say \A has \emph{LUB property} if
    $$\forall A_0\subset A,A\not=\emptyset\land\ex \text{upper bound for }
    A_0 \RA \ex \text{lub$\{A_0\}$}\in A$$
  \end{definition}
  \begin{example}
    $A=(-1,1)$. \eg $X=\set{1-\frac{1}{n}\mid n\in\ZZ^+}$ does not have an
    upper bound, thus vacuously true. $\set{-\frac{1}{n}\mid
      n\in\ZZ^+}$ has upper bound of any number in $[0,1)\subset A$, and
    $\operatorname{lub}(X)=0\in (-1,1)$.
  \end{example}
  \begin{example}
    Counterexample. $A=(-1,0)\cup(0,1)$. $\set{-\frac{1}{n}\mid n\in\ZZ^+}$ has
    upper bound of any $(0,1)\subset A$, while
    $\operatorname{lub}(X)=0\not\in A$.
  \end{example}
  \begin{remark}
    The completeness property of $\RR$ as an axiom derives this property.
  \end{remark}

  \begin{property}{$\RR$ field}

    \textbf{Algebraic properties}
    \begin{enumerate}
    \item assoc: $(x+y)+z=x+(y+z); (xy)z=x(yz)$
    \item comm: $x+y=y+x; xy=yx$
    \item id: $\ex!0, x+0=x$; $\ex!1,x\not=0\RA x1=x$
    \item inv: $\forall x,\ex!y,x+y=0$; $\forall x\not=0,\ex!y,xy=1$
    \item distr: $x(y+z)=xy+xz$
    \end{enumerate}

    \textbf{Mixed algebraic and order property}
    \begin{enumerate}[resume]
    \item $x>y\RA x+z>y+z$; $x>y\land z>0\RA xz>yz$
    \end{enumerate}

    \textbf{Order properties}
    \begin{enumerate}[resume]
    \item $<$ has LUB property
    \item $\forall x<y,\ex z, x<z\land z<y$
    \end{enumerate}
  \end{property}

  1-6 make $\RR$ a field. 1-6 + 7 make $\RR$ an ordered field. 7-8
  makes $\RR$, called by topologists, a \defn{linear continuum}.
\end{chapter}

\end{document}

%%% Local Variables:
%%% mode: latex
%%% TeX-master: t
%%% End:
