% -*- TeX-master: "master.tex" -*-

\setcounter{section}{9}
\setcounter{subsection}{0}
\subsection{Natural Transformations}

Natural transformation is the third thing in the most important foundations of
category theory. The first is the definition of category, and the second is
functor, natural transformation is the third.

Category is about structure. Functors are mappings between categories that
preserves the structure. The intuition about functor is that they take a
category and embed it into another category. It's like searching a pattern in a
category. We would also want to compare two functors, or said differently, images
given by functors. So compare things by mapping one into another, that's a
natural way of modeling comparsion.

Natural transformations are defined as structure-preserving mappings between
functors. Given two categories, $C$ and $D$, and two functors $F$ and $G$ which
$F,G: C\to D$, let $a$ be an object in $C$, which $F$ maps $a$ to $F a$, $G$
maps $a$ to $G a$. If we want to map one functor to another functor, we want to
map $F a$ to $G a$. We already have morphisms in $D$ for mapping $F a$ to $G a$.
To define mapping between functors on $a$ is the same as picking a morphism in
$D$ that maps from $F a$ to $G a$. So a natural transformation will, for every
object $a$ in $C$, pick a morphism in $\Hom_D(F a, G a)$. This way a natural
transformation is creating a family of morphisms in $D$. These individual
morphisms are called \emph{components} of the natural transformation. Denoted as
$\alpha_a$ where $\alpha$ is the natural transformation.

So far, we only talked about what a natural transformation does on objects, we
also need to take a look on morphisms. Given objects $a, b$ in $C$ and morphism
$f: a \to b$, we have the functors $F, G$ which map $a$ to $F a$, $G a$, and map
$b$ to $F b$, $G b$, respectively. Let natural transform $\alpha$ on $a$ and
$b$ be $\alpha_a: F a\to G a$ and $\alpha_b: F b \to G b$.

\begin{figure}[h]
\centering
\begin{tikzcd}[math mode=true, sep=large]
  a \ar{rrr}{F}\ar{rrrdd}[near start]{G}\ar[green]{d}{f} & & & F
  a\ar[red]{dd}[near start]{\alpha_a}\ar[green]{rd}{F f} & \\
  b \ar[crossing over]{rrrr}{F}\ar{rrrrdd}{G} & & & & F b\ar[red]{dd}{\alpha_b} \\
  & & & G a\ar[green]{rd}{G f} & \\
  & & & & G b
\end{tikzcd}
\caption{natural transofrm}
\end{figure}

We need to have the natural transform preserve structures of $F$ and $G$. That
means these two paths from $F a$ to $G b$ must be equal. In other words, the
diagram must commute.

Naturality condition is very strong

Example in set. Natruality is almost determined.
