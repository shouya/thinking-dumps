% -*- TeX-master: "master.tex" -*-

\setcounter{section}{8}
\setcounter{subsection}{1}
\subsection{Function objects, Exponentials}

\subsubsection{Function objects}

Functions are separated from types so far: we learned in Hask that types are
objects and functions are morphisms. For category $\Set$, objects are all the
sets, and functions are mapping between the objects. Functions from object $a$
to object $b$ forms a set, called hom-set, denoted as $\Hom(a,b)$. And because
$\Set$ contains all sets, this morphisms representing hom-set $\Hom(a,b)$ is also
an object in $\Set$.

In an arbitrary category, we don't have the hom-sets as internal objects; but we
can view them as external things being in other categories. What we want is
something we can represent these hom-sets inside the category. It is indeed
possible to define these objects in many categories, although not in arbitrary
category. Just like we define product and coproduct, the same way we can apply
to define this function object. We can use univeral construction to define
function object.

\begin{remark} The 3 steps towards universal construction:
\begin{enumerate}
\item Define a pattern.
\item Define ranking between matches.
\item Find the best one.
\end{enumerate}
\end{remark}

It's clear to see that the pattern will involve three objects, the function $z$,
the argument type $a$ and the return type $b$. In a category, we can't directly
say $z$ takes argument $a$ returns $b$, so we think how can we work on this. For
example, in category $\Set$, we can say take $z$ and $a$ to form a pair $(z,a)$
such that $(z,a) \iso b$. This pairing in $\Set$ represent a cartesian product
of $z$ and $a$. This perfectly captures the idea of function application, and
this pattern can be generalize to any category.

This pattern implies that in order to define this construction in a category,
the category must contain products. This makes more sense when later we discuss
function objects as exponentials.

The pattern, drawn in communitive diagram, looks like:

\begin{center}
  \begin{tikzcd}[sep=large]
  z \rar[dash,dashed] & z\times a \rar[swap]{g} & b  \\
  & a \uar[dash, dashed] &
  \end{tikzcd}
\end{center}

Note that the morphism $g$ is called a \emph{eval} morphism. Next we need to
define a ranking to select the best function $z$ from $a$ to $b$.

\newcommand{\arb}{\ensuremath{}(a\Rightarrow{}b)}

With the pattern above, we say $\arb$ is the function object from $a$ to $b$ if
for other patterns defined on $z$ and $a$, we have a unique morphism $h:
\cd{z\rar[dashed] \& \arb}$, that means we have to be able to factorize
$g:\cd{z\times a\rar\& b}$ into $\textrm{eval}$ arrow and a unique arrow
$\cd{z\times a\rar[dashed] \&\arb\times a}$. $\textrm{eval}$ arrow is given by
the pattern, so we need to find the other arrow. By previously learned
functoriality of ADTs, we know that product is a bifunctor, thus it not only
maps objects, it also maps morphisms. By which we can say the unique arrow we
are looking for is $h\times \id_a$. The whole construct can be shown as the
diagram below:

\begin{center}
  \begin{tikzcd}[sep=large]
    z  \dar[dashed]{h} &
    z\times a \ar{d}[dashed]{h\times \id_a} \ar[rd, start anchor={east}, bend left=30, "g"] & \\

    (a\Rightarrow b)  &
    (a\Rightarrow b)\times a \ar[r,swap,start anchor={east},"\textrm{eval}"] & b  \\

    & a  &
  \end{tikzcd}
  \\
  (arrows representing $\pi$ maps in products are omitted for clarity)
\end{center}


In short, the function object $\arb$ only exists if for every possible $z$ and
$g$ in that pattern, there is a unique $h$ from $z$ to $\arb$ such that:

\[
  g = \textrm{eval} \circ (h \times \id)
\]

In most languages, functions with two arguments are basically functions with one
argument which is a pair. Think of $g$ as such a function, that takes $z$ and
$a$ as arguments. Our definition of function objects implies given $z$ and $g$
we actually have a unique arrow $h$, that takes a $z$ and returns a function
object $\arb$.

This captures the idea of partially applying function is
equivalent to the applying function with its arguments given in a pair; the idea
can be generalized to functions with multiple arguments. This gives rise to the
concept of currying.

\begin{lstlisting}
h :: z -> (a -> b)
g :: (z, a) -> b

curry :: ((a,b)->c) -> (a->b->c)
curry f = \lambdaa->(\lambdab->f (a,b))

uncurry :: (a->b->c) -> ((a,b)->c)
uncurry f = \lambda(a,b)->(f a) b
\end{lstlisting}


\subsubsection{Exponential}

Function objects in category theory are actually called exponential. A function
object $\arb$ will be called $b^a$. This makes sense if you think of the
cardinality of function types. In Haskell, $\Bool \to \Int$ is basically $(\Int,
\Int)$, or $\Int \times \Int$ in category notation, or simpler $\Int^2$.
Remember we learned the correspondence between nature numbers and Hask types:
$1$ corresponds to unit $()$, $2$ corresponds to $\Bool$, etc. Then $\Int^2$ is
basically $\Int^\Bool$.

The idea shows the connection between products and exponentials (\ie.
functions). A special kind of categories called \textbf{Cartesian Closed
  Category} (CCC) is useful in programming. Cartesian means there is a product
for every pair of objects. Closed means the products are \emph{inside} the
category, which means it has all the exponentials objects as well. Also, it must
have the terminal object. Why? Terminal object is like the $0$-th power of any
objects. Sometimes we also want co-products in the category, this kind of
categories is called Bi-Cartesian Closed Categories (BCCC), which also has the
initial object.

With products and co-products both being monoidal, we can combine them together
to form a semi-ring, and do algebra on them. Now we add exponential to the
system. With exponentials we can do more algebra, for example, $a^0~=~1$.

We can interpret $a^0 = 1$ as $\Void \to a \sim ()$, \ie. a function from void
type to $a$ is equivalent to the unit type. We need to show the function on the
left hand side exists and is unique. Turns out there is this function, called
\textbf{absurd}.

What about $1^a=1$? It is a function takes $a$ and returns unit $()$; we say it
is equivalent to just $()$. There is only one such function indeed, called
$\texttt{const}~()$.

Let's try another one, $a^1=a$, This is saying a function from $()$ to $a$ is
equivalent to $a$. These functions are just the selecting functions; there are
as many as the number of elements in $a$.
