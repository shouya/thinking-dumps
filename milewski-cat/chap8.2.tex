% -*- TeX-master: "master.tex" -*-

\setcounter{section}{8}
\setcounter{subsection}{1}
\subsection{Type algebra, Curry-Howard-Lambek isomorphism}

\subsubsection{Type algebra}

We have learned the correspondence between some categorical structures and
algebra.

\begin{figure}[h]
\centering
\begin{tabular}{ll}
\hline
Algebra      & Categorical Structure         \\ \hline
$a + b$      & Coproduct                     \\
$a \times b$ & Product                       \\
$a^b$        & Exponential (Function Object) \\ \hline
\end{tabular}
\caption{Categorical representation of algebra operations}
\end{figure}

Now we figure out how to do algebra in the category of types. Look at this
algebra expression,

\[
  a^{b+c} = a^b \times a^c
\]

This is essentially saying that \code{Either b c -> a} is equivalent to a pair
of functions \code{(b -> a, c -> a)}. We can do case analysis: Either type gives
one of its params, if given parameter is \code{b}, it should call function
\code{b -> a}; if given \code{c}, it should call \code{c -> a}. It works. It's
easy to show the other way around from \code{(b -> a, c -> a)} to \code{Either b
  c -> a}.

Let's see the next one,

\[
  (a^b)^c = a^{b\times c}
\]

It should be easy to recognize this pattern is exactly what currying does, that
is, \code{c -> (b -> a)} is equivalent to \code{(b, c) -> a}.

Next one,

\[
  (a\times b)^c = a^c\times b^c
\]

This is saying \code{c->(a,b)} is equivalent to \code{(c -> a, c -> b)}, which
makes sense since given a function that takes a \code{c} and spit out \code{a}
and \code{b} is the same as given two function, both take a \code{c}, one spit
out \code{a} and another one spit out \code{b}.

All above algebra holds in Haskell and other BCCC languages.


\subsubsection{Curry-Howard-Lambek isomorphism}

We started to realize that the algebra we learned in high school seems to apply
to a broader and a broader range of things. Starting from numbers, then to variables,
then to functions, then to types and categories. All these things are so similar
and the similarity extends further.

The other thing fall out from this is that the same structures here we see in
types and categories appear in logics as well. This is the famous
\textbf{Curry-Howard isomorphism}, or \textbf{Proposition as Types}. So all
these we talked about in type theory and categories today actually have
one-to-one correspondence in logic.

The isomorphism between type theory and logic starts with identifying what it
means to be a proposition. A proposition is a statement that can either be true
or false. In type theory, a type can be either inhabited or not. The truth of a
proposition corresponds to that the type has elements, that is, they are
inhabited.

Most of the types we learned are inhabited, while there are some that are not,
so they corresponds to false propositions. A classic example we know is the
\code{Void} type, which has no members.

Another thing we have in logic is proof. If we want to prove a proposition we
can construct an element of the corresponding type. In logic there are two basic
propositions true and false, true is always true, and false is always false. The
corresponding things in type is: \code{Void} type corresponds to false,
\code{Unit} type corresponds to true.

Then we have conjunction ($\land$). The conjunction of $a$ and $b$ ($a\land b$)
is corresponds to the pair \code{(a,b)} in type theory. We can only form the
pair type if only we have both an element from $a$ and an element from $b$.

Disjunction ($\lor$) works similarly, in the way that $a \lor b$ corresponds to
\code{Either a b}. If we have either an element of $a$ or an element of $b$, we
can construct a \code{Either a b}.

Implication ($\Rightarrow$) corresponds to the function type. $a \Rightarrow b$
corresponds to the function type \code{a -> b}. The function type \code{a -> b}
is essentially meaning, give me an element of \code{a}, I'll give you an element
of \code{b}. In other words, if I have the function then I can produce \code{b}
from \code{a}. The same is true for implication. In logic, if we have
$(a\Rightarrow b) \land a$ we can conclude $b$. Likewise, if we have an
inhibited pair \code{(a -> b, b)}, we can derive an element of \code{b}.


\begin{figure}[h!]
\centering
\begin{tabular}{lll}
\hline
Category        & Logic            & Type              \\\hline
terminal object & true             & \code{()}         \\
initial object  & false            & \code{Void}       \\
product         & $a\land b$       & \code{(a,b)}      \\
coproduct       & $a\lor b$        & \code{Either a b} \\
exponetial      & $a\Rightarrow b$ & \code{a -> b}     \\
\hline
\end{tabular}
\caption{Table showing the correspondance}
\end{figure}

Curry-Howard proposed the correpondence between logic and type theory, and
Lambek contributed the correspondence between them and category theory.

A cartesian closed category is a model for logic and also a model for type
theory, where all these three things are the same. Some mathematicians may
discover some idea in logic, and then computer scientist can pick it up and find
it can be some type. Some modern concepts are discovered and found applicable in
different fields in this way. An example is Linear Logic, which computer
scientists translated into programming languages using Curry-Howard
Correspondence, called Linear Type. In a linear type system each object is used
exactly once. This system makes up the basis of Rust's ownership and C++'s
\code{unique_ptr}.
