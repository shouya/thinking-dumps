\setcounter{section}{6}
\setcounter{subsection}{0}
\subsection{Functors}

Universal contruction is about to pick the ``best'' embodiment of an idea. \eg.
there is a lot of product in the universe, so we pick the ``best'' one, the one
that every product gets factorized into. Thus comes the ``universal''
construction.

Functors are just mapping between categories.

Why so important? We look for structures/patterns in categories and try to
emphasize/extract the pattern from them. Functor is used to find these patterns
and map them into a category when we can recognize these patterns.

``Category is the definition of structure''. In orther words, ``to recognize a
certain structure in a category'' is the same as defining the pattern as a
category. Being able to recognize a category inside another category is just
doing pattern recognition.

The mappings we are interested in are those which preserve structures. A
function is a mapping from set to set, which does not have a structure. A single
set represented in a category is just a category of bunches of objects and with
no arrows except the identities. No structure. (Discrete category)
\\

\begin{definition}{Functor}
  $F$ is a map between categories that preserves the structure.
  \ie. it maps arrow to arrow.
  For categories $C$, $D$, a functor is that for every
  $f \in C(a, b)$, $F$ maps it into $F~f \in D(F~a, F~b)$ that preserves the
  structure.

\[
\begin{tikzcd}[sep=huge]
  a \rar{F_a} \dar[red]{f} & F~a \dar[red]{F~f} \\
  b \rar{F_b} & F~b
\end{tikzcd}
\]

  Since a hom-set is a set, a functor just defines this ``huge'' function.

  Preserving the structure means for $g \circ f \in C$, $F (g \circ f) = (F g)
  \circ (F f) \in D$.

  Naturally one have to also make sure
  $\forall a \in C, F(id_a) = id_{F a} \in D$.
\end{definition}

\subsubsection{Functor properties}

\begin{definition}{faithful/full}
Functors that don't collapse structure is called ``faithful''. A faithful
functor is injective on \emph{hom-sets}. Correspondingly, A functor is ``full'' when it
is surjective. Functors that are ``full'' or ``faithful'' are only about
injective/surjective on the arrows, not about objects. Functor could map two
distinct objects
$a, b\in C$ into some $c \in D$, as long as the arrows between $a$ and $b$
doesn't map to the same arrow in $D$, this functor could still be faithful.
\end{definition}

\newcommand{\singletoncat}{%
\begin{tikzcd}[cramped]
  \cdot \ar[loop]{r}{id}
\end{tikzcd}%
}

\subsubsection{Interesting functors}

\begin{definition}{Selecting functor}
The possible functor mapping from category 1 (singleton category) to
another category $C$ is unique (up to iso), that is, to map $\id \in \mathrm{Arr}(1)$ arrow
to the $\id$ arrow on some
object in $D$. This process is like ``selecting'' an element in $D$.
\end{definition}
\hfill
\begin{definition}{Constant functor}
Another important functor mapping from $C$ to $D$ is called constant functor, which maps
all arrows in $C$ into the $\id$ arrow of an object $c \in D$. Constant functor
on object $c$ is denoted as $\Delta_c$.
\end{definition}
\hfill
\begin{definition}{Endofunctor}
  An endofunctor is a functor that maps from $C$ to itself, $F: C \to C$. Haskell
  functors are all endofunctors mapping from and to the $\Hask$ category.
  Haskell functors consists of two parts:
  \begin{enumerate}
  \item Mapping between types (types are objects in $\Hask$), \ie. Type constructors.
  \item Mapping between functions, \ie. \verb+fmap+.
  \end{enumerate}
\end{definition}
\hfill
\begin{remark}
  Take this example,
  \begin{lstlisting}
fmap :: (a -> b) -> (Maybe a -> Maybe b)
fmap f Nothing = Nothing *'\label{nothing}'*
fmap f (Just x) = Just (f x)
  \end{lstlisting}

  To haskell, which uses parametrically polymorphism, line \ref{nothing}
  is the only possible implementation for \lstinline{fmap f Nothing}. However,
  it's not true to math. For languages which doesn't use parametrically
  polymorphism, it might be possible to return other values. \eg. Return
  \lstinline{Just 0} for if type \lstinline{a} is \lstinline{Int}. Haskell is
  imposing a stronger condition to be a functor.

  \newcommand{\Maybe}{\lstinline{Maybe}}
  Let's check if \Maybe~really is a functor. Questions to ask:
  \begin{enumerate}
  \item Does \lstinline{fmap} preserves identity?
  \item Does \lstinline{fmap} preserves composition?
  \end{enumerate}

  Surely it does.
\end{remark}
