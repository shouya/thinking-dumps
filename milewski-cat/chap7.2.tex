% -*- TeX-master: "master.tex" -*-

\setcounter{section}{7}
\setcounter{subsection}{1}
\subsection{Monoidal Categories, Functoriality of ADTs, Profunctors}

\subsubsection{Monoidal Categories}

We start out by seeing how we can generalize monoid in a category. We know
pretty well how monoid based on set theory, we got a unit $\mathbf{1} \in \Set$,
a binary function $f : \Set \times \Set \to \Set$, few rules and so on.

In monoid categories, we need to think about what it means to multiply two
object. This gives rise to product category. The monoidal unit in this category
is the unit object, the singleton set in $\Set$, or the terminal object to be
more general.

Here's a how we the terminal object is a good candidate for the unit. This is to
say, in a product category, $(a, ()) \iso a$.

\begin{center}
  \begin{tikzcd}[sep=large]
    & b
    \ar[bend right=15]{ldd}[left]{p}
    \dar[dashed]{p}
    \ar[bend left=15,dashed]{rdd}{\mathrm{unit}_b} & \\
    & a \ar{ld}{\id_a} \ar[dashed]{rd}[swap]{\mathrm{unit}_a} & \\
    a & & ()
  \end{tikzcd}
\end{center}

As we can see, the projections $\pi_1 = \id$ and $\pi_2 = \mathrm{unit}$. Given
any $\cd{b \rar{p} \& a}$, we have $\mathrm{unit}_b$ which maps $b$ to $()$, we
always have a unique factorization $\cd{b \ar[dashed]{r}{p} \& a}$. This means
the projections just worked and we indeed have $(a, ()) \iso a$.

So in a monoidal category (at least product category), monoid unit is the
terminal object. This also apply to other categories. Say the coproduct
category. In this case the terminal object is the empty set (initial object of
the coproduct category). And they both are specialized bifunctor category. All
these things as product, coproduct, any bifunctors are called \emph{tensor product}
(denoted $C \otimes C \to C$).

What we really need is to have a bifuntor, and a unit for the bifunctor. Then we
can get a monoidal category, assuming monoidal properties hold.

\subsubsection{Functoriality of ADTs}

Dr. Milewski has mentioned that ADTs has functoriality by their nature many
times in his lecture.

From the previous section, we see how these monoidal categories are related to
their corresponding bifunctors. We see, product types are functorial, sum
types are functorial, what about other types, \code{Bool}, \code{Int}?

So we still have some types that do not depend on type variables, how can we
make them instance of \code{Functor}? We can say they dependent on type indeed,
but in a \emph{trivial} way. So here's the place for the constant functor trick.

\begin{lstlisting}
data Const c a = Const c

instance Functor (Const c) where
  fmap :: (a -> b) -> Const c a -> Const c b
  fmap f (Const c) = Const c
\end{lstlisting}

In category theory, constant functor $\Delta_c$ maps every object in a category
to the object $c$ and every morphisms to $\id_c$. Taking it to the Haskell world
where every thing is in $\Hask$, the constant functor is used to map everything
into the given value. Think of functors working as containers, then the constant
functor is like an empty container. It eats up all information and collapse into
something that does not contain any inforamtion.

Another trivial functor is the identity functor:

\begin{lstlisting}
data Identity a = Identity a

instance Functor (Identity a) where
  fmap :: (a -> b) -> Identity a -> Identity a
  fmap f (Identity a) = Identity (f a)
\end{lstlisting}

This functor maps everything back to itself. So this functor is like a container
with one thing in it. So this is how we convert any ADTs into our standard
notion.

\begin{lstlisting}
data Maybe a = Nothing | Just a
-- sum types can be decomposed into Either
data Maybe a = Either Nothing (Just a)
-- any constant type can be represented using Const functor
-- and Just is just an Identity functor
data Maybe a = Either (Const () a) (Identity a)
\end{lstlisting}

Now we have a \code{Const ()} functor and an \code{Identity} functor; and
\code{Either} is just a bifunctor. Thus overall \code{Maybe} is a bifunctor.
Since we have the same parameter \code{a} for \code{Const ()} and
\code{Identity}, we can add a diagonal functor before the bifunctor of the
\code{Either}/\code{Maybe}, reducing it into a regular endofunctor.

This process of functorizing any ADT type can be automated, and it's built-in to
Haskell as an extension: \code{DeriveFunctor}. Could there be more than one
\code{fmap} defined for an ADT functor? No, Because of parametricity of
polymorphism in Haskell, the canonical one is the only possible one!

\hfill
\begin{remark}
  A \emph{diagonal functor} maps $C \to C \times C$ by replicating everything
  including objects and morphisms $a \mapsto (a, a)$.
\end{remark}

\subsubsection{Profunctors}

Other than product (\code{()})and sum types (\code{Either}). Another type
constructor that takes two types is the function type ($\to$). We can write
\code{(}$\to$\code{)}~\code{a b} to represent \code{a -> b}. The question is, is
$\to$ a bifunctor?

We already knew that \code{(-> c)} is a functor, called the \emph{Reader
  functor}.

\begin{lstlisting}
instance Functor (-> c) where
  fmap :: (a -> b) -> (c -> a) -> (c -> b)
  fmap f g = f  .  g
\end{lstlisting}

What about we fix on the returning type and see if that's a functor. The
\code{fmap} would have type \code{fmap :: (a -> b) -> (a -> c) -> (b -> c)}.
If we name $f: a\to b$ and $g: a \to c$, we have this diagram: \cd{a \rar{f}
  \dar{g} \& b \ar[red]{ld}{} \\ c \& }. We need to find the red arrow and
obviously it cannot be constructed using $f$ and $g$, so it is not a functor.

Or is it? Why not saying it a functor mapping from the opposite category? This
kind of functors is called a \emph{contravariant functor}. Using the container
analogy, a contravariant functor is like a container that store negative things:
instead of containing something we need to give it something.

\hfill
\begin{remark}
  The \emph{opposite category} $C^{op}$ of category $C$ is a category where for
  every objects in $C$ there is an object in $C^{op}$ and for every morphisms $f
  : a\to b$ in $C$ there is a morphism $f^{op} : b \to a$ in $C^{op}$.
\end{remark}

\begin{lstlisting}
class Contravariant f where
  contramap :: (b -> a) -> f a -> f b

data Op c a = Op (a -> c)

instance Contravariant (Co c) where
  contramap :: (b -> a) -> (Op c a) -> (Op c b)
--contramap ::  (b -> a) -> (a -> c) -> (b -> c)
  contramap f (Op g) = Op (f  .   g)
\end{lstlisting}

So a profunctor is just a bifunctor that is contravariant in the first argument
and covariant in the second. What's the problem?
\footnote{\url{https://www.schoolofhaskell.com/school/to-infinity-and-beyond/pick-of-the-week/profunctors\#profunctors}}

\begin{lstlisting}
class Profunctor p where
-- define dimap
  dimap :: (a' -> a) -> (b -> b') -> p a b -> p a' b'
  dimap f g h = lmap f (rmap g h)

-- or both lmap and rmap
  lmap :: (a' -> a) -> p a b -> p a' b
  lmap = (`dimap` id)
  rmap :: (b -> b') -> p a b -> p a b'
  rmap = (id `dimap`)
\end{lstlisting}

To function type, let's see how it is implemented. $f : a' \to a$, $g : b \to
b'$, and $h : a \to b$. So the diagram is just \cd{a' \rar{f} \ar[red,bend
  right=90]{rrr}{} \& a \rar{h} \& b \rar{g} \& b'}. Bravo, finally we have this
approachable red arrow.

The functions \code{lmap} and \code{rmap} correspond to \code{contramap} and
\code{fmap}, which is a implication of the fact that a profunctor is a bifunctor
over a contravariant functor and a covariant functor.
