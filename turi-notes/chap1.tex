\section{Universal property}

\subsection{Natural numbers in set theory and category theory}

\textbf{A1} is the regular Peano's definition of natural numbers. There is
nothing new.

\textbf{A2} is more interesting to investigate. It defines natural
number set ($N$) to be the ``initial'' object of the category of all
natural-number-like sets ($X$). The essential part of a univeral
property is the unique arrow, or
\emph{factorization}. In our case, it is the $f : X \to N$. Here is an
example of an $X$ in \textbf{A2}.

\begin{example}
  $(-1) \in X \rTo^g X$ where
  $g := a \rMapsto a-1$. With this case
  $f := a \rMapsto -(a+1)$.
\end{example}

This is straightfoward, just a demonstration of what is it about.


\begin{quotation}
  The \emph{Recursion Theorem}
  \footnote{\url{https://en.wikipedia.org/wiki/Recursion#The_recursion_theorem}}
  guarantees recursively defined functions exists. Given a set $X$, an
  element of $e\in X$ and a function $g : X \to X$, the theorem states
  there is a unique function $f : N \to X$, such that
  \begin{align}
    f(0) &= e \\
    f(n+1) &= g(f(n))
  \end{align}
\end{quotation}

This is essentially defines a factorization from $N$ to $X$.

So then the proof of \textbf{A1} and \textbf{A2} are isomorphic.


%%% Local Variables:
%%% mode: latex
%%% TeX-master: "master.tex"
%%% End:
